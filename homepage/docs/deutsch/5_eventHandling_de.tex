\chapter{Event Handling}
\label{sec:EventHandling}

\paragraph{Beweggrund:}
Ereignisse(\emph{Events}) brauchen in {\sl Flash} sehr viel Rechenleistung und sind ein grundlegender Bestandteil bei der Entwicklung von Benutzeroberfl�chen. Viele Entwickler verwenden den in {\sl Flash} inkludierten \emph{AsBroadcaster}\footnote{Nicht dokumentiertes Feature von Flash MX 2004} oder die \emph{EventDispatcher}\footnote{mx.events.EventDispatcher} Klasse von {\sl Macromedia} f�r diese Problematik. Jedoch gibt es weder genaue Spezifikationen f�r \emph{EventListener} noch f�r einzelne Events und deren �bergabewerte.
Die Festlegung von Schnittstellen �ber Interfaces ist f�r eine "`saubere"' Programmierung unumg�nglich. Es fehlen unter anderem f�r Entwickler von \emph{Listenern}(Objekte die auf bestimmte Events warten) wichtige Informationen bez�glich Events und der �bergabeparameter.

\paragraph{L�sungsansatz:}

Um zu einer L�sung zu gelangen m�ssen folgende Probleme bew�ltigt werden:

\begin{itemize}
	\item Der Objektentwickler muss definieren, welche Events abgewartet werden k�nnen.
	\item Der Listenerentwickler muss alle Events definieren.
	\item Der Objektentwickler muss die M�glichkeit haben, Events auszul�sen.
	\item Der Objektentwickler sollte genauere Informationen einem Event hinzuf�gen k�nnen.
\end{itemize}
Die {\sl as2lib} unterst�tzt \emph{Event Handling}, da es ein Kernst�ck der Applikationsentwicklung ist. Verwendet man direkt den Flash internen \emph{AsBroadcaster} kann es zu einer ineffizienten Implementierung kommen. Der \emph{EventDispatcher} von {\sl Macromedia} ist nicht frei zug�nglich, kann also nur verwendet werden wenn man {\sl Macromedia Flash} k�uflich erworben hat, und stellt nicht alle ben�tigten Funktionalit�ten zur Verf�gung.

Das wichtigste Interface des \emph{event} Packages mit der man als Entwickler in Ber�hrung kommt ist der \emph{EventBroadcaster}\footnote{org.as2lib.env.event.EventBroadcaster}. Es k�nnen mehrere Listener(Zuh�rer) zum EventBroadcaster(Ereignissverteiler) hinzugef�gt,
\begin{lstlisting}[frame=single]
addListener(listener:EventListener)
\end{lstlisting}
aber auch wieder entfernt werden.
\begin{lstlisting}[frame=single]
removeListener(listener:EventListener)
\end{lstlisting}
Wie man in den Methodenaufrufen erkennen kann m�ssen die Listener Objekte das EventListener Interface implementieren. F�r eigene Projekte empfiehlt es sich ein eigenes EventListener Interface zu erstellen.

M�chte man den \emph{EventBroadcaster} verwenden kann man seine Funktionalit�ten einfach nach folgender Methode
erwerben, siehe Abb.\ \ref{fig:as2libeventhandling}, S.\ \pageref{fig:as2libeventhandling}. Diese Abbildung zeigte eine einfach Anwendung des {\sl as2lib} Event Handlings. Es werden bis auf den SimpleEventListener nur Klassen der {\sl as2lib} verwendet.

\begin{lstlisting}[frame=single][caption=SimpleEventListener]
import org.as2lib.env.event.EventListener;
import org.as2lib.core.BasicClass;

class edu.diplomarbeit.listener.SimpleEventListener 
	extends BasicClass implements EventListener {
	
	public function onTest(){
		trace("onTest");
	}
}
\end{lstlisting}

Die einzelnen Schritte des as2lib EventHandlings werden durch die nummerierten Pfeile dargestellt.

\begin{figure}
\begin{center}
\includegraphics{eventhandling.eps}
\caption{Ablauf eines einfachen Event Handling Beispiels.}
\label{fig:as2libeventhandling}
\end{center}
\end{figure}

\chapter{Test Cases}
\label{sec:TestCases}

Ein \emph{Test Case}(Testfall) ist eine Klasse/Methode, die eine bestimmte Klasse/Methode auf ihre korrekte Funktionsweise �berpr�ft. Jeder Entwickler muss f�r seine Klassen Testf�lle erstellen und wird von einem Test Case System unterst�tzt, das Funktionalit�ten f�r ein automatisches Testen zur Verf�gung stellt.

\paragraph{Beweggrund:}
Obwohl bereits ein Testsystem f�r {\sl Actionscript 2} Klassen existiert, as2unit\footnote{\href{http://www.as2unit.org}{www.as2unit.org}}, wurde aus folgenden Gr�nden ein eigenes Testsystem erstellt:

\begin{itemize}
	\item {\sl as2unit} unterst�tzt nur wenige Funktionalit�ten({\sl as2unit} \textbf{7} Testmethoden -- {\sl as2lib} \textbf{15} Testmethoden).
	\item {\sl as2unit} ist trotz verk�ndeter Offenlegung des Quellcodes immer noch nicht Open Source.
	\item Es gibt immer noch keine offizielle Dokumentation der {\sl as2unit}.
	\item {\sl as2unit} ist nur als Komponente vorhanden.
	\item {\sl as2unit} kann immer nur eine Klasse testen.
	\item {\sl as2unit} erlaubt eine Ausgabe nur �ber trace.
\end{itemize}

\paragraph{L�sungsansatz:}

Um Testf�lle durchf�hren zu k�nnen, sollen nicht zwei unterschiedliche Aktionen notwendig sein(Einbinden der Flashkomponente, setzen der Parameter ), wie es momentan bei der {\sl as2unit} der Fall ist. Es soll dem Entwickler m�glich sein einen Testfall durch einen einfachen Methodenaufruf durchf�hren zu k�nnen. Sowohl ein direkter Aufruf einer Klasse, als auch ein Aufruf eines gesamten Packages, falls mehrere Test F�lle durchzuf�hren sind, ist m�glich.
\begin{lstlisting}[frame=single]
import org.as2lib.test.unit.Test;

// Hier Testf�lle einf�gen.
test.org.as2lib.core.TReflections;
test.org.as2lib.util.TReflectUtil;
test.org.as2lib.util.TStringUtil;

Test.run("test.org");
\end{lstlisting}

Die zu testenden F�lle m�ssen vor dem Testaufruf angef�hrt werden, damit sie zur Laufzeit verf�gbar sind.
�hnlich wie bei Test Case Systemen zB.: JUnit\footnote{http://www.junit.org} stehen dem Entwickler in der {\sl as2lib} bereits eine Vielzahl von Methoden zu Verf�gung(optionale Parameter sind durch [] gekennzeichnet):

\begin{itemize}
	\item \textbf{assertTrue}([\textit{message:String}], \textit{testVar1:Boolean}) - Ist testVar1 false wird eine Fehlermeldung ausgegeben.
	\item \textbf{assertFalse}([\textit{message:String}],\textit{ testVar1:Boolean}) - Ist testVar1 true wird eine Fehlermeldung ausgegeben.
	\item \textbf{assertEquals}([\textit{message:String}], \textit{testVar1}, \textit{testVar2}) - Sind die �bergebenen Parameter nicht gleich wird eine Fehlermeldung ausgegeben.
	\item \textbf{assertNotEquals}([\textit{message:String}], \textit{testVar1}, \textit{testVar2}) - Sind die �bergebenen Parameter gleich wird eine Fehlermeldung ausgegeben.
	\item \textbf{assertNull}([\textit{message:String}], \textit{testVar1}) - Ist testVar1 nicht null wird eine Fehlermeldung ausgegeben.
	\item \textbf{assertNotNull}([\textit{message:String}], \textit{testVar1}) - Ist testVar1 gleich null wird eine Fehlermeldung ausgegeben.
	\item \textbf{assertUndefined}([\textit{message:String}], \textit{testVar1}) - Ist testVar1 nicht undefined wird eine Fehlermeldung ausgegeben.
	\item \textbf{assertNotUndefined}([\textit{message:String}], \textit{testVar1}) - Ist testVar1 gleich undefined wird eine Fehlermeldung ausgegeben.
	\item \textbf{assertIsEmpty}([\textit{message:String}], \textit{testVar1}) - Ist testVar1 weder undefined noch null wird eine Fehlermeldung ausgegeben.
	\item \textbf{assertIsNotEmpty}([\textit{message:String}], \textit{testVar1}) - Ist testVar1 undefined oder null wird eine Fehlermeldung ausgegeben.
	\item \textbf{assertInifinity}([\textit{message:String}], \textit{testVar1}) - Ist testVar1 nicht unendlich(d.h. != Infinity) wird eine Fehlermeldung ausgegeben.
	\item \textbf{assertNotInfinity}([\textit{message:String}], \textit{testVar1}) - Ist testVar1 unendlich(d.h. == Infinity) wird eine Fehlermeldung ausgegeben.
	\item \textbf{fail}(\textit{message:String}) - F�gt eine selbst definierte Fehlermeldung zu allen Error-Ausgaben hinzu.
	\item \textbf{assertThrows}(\textit{exception:Function, atObject, theFunction:String, parameter:Array}) - Wird beim Ausf�hren der �bergebenen Funktion(theFunction) des Objektes(atObjekt) mit den �bergabeparametern(parameter) keine Exception geworfen wird eine Fehlermeldung ausgegeben.
	\item \textbf{assertNotThrows}(\textit{exception:Function, atObject, theFunction:String, parameter:Array}) - Wird beim Ausf�hren der �bergebenen Funktion(theFunction) des Objektes(atObjekt) mit den �bergabeparametern(parameter) eine Exception geworfen wird eine Fehlermeldung ausgegeben.
\end{itemize}

Ein Testfall muss von der \emph{Test} Klasse\footnote{org.as2lib.test.unit.Test} abgeleitet werden. Jede Methode des Testfalles, die mit "`test"' beginnt, wird von dem Testsystem aufgerufen. Wie genau ein erstellter Testfall aussehen k�nnte wir in folgendem Codebeispiel gezeigt:
\clearpage
\begin{lstlisting}[frame=single]
import org.as2lib.test.unit.Test;
import org.as2lib.core.BasicClass;
import org.as2lib.env.reflect.ClassInfo;
 
class test.org.as2lib.core.TReflections extends Test {
	private var clazz:BasicClass;
	
	public function TReflections(Void) {
			clazz = new BasicClass();
	}
	
	public function testGetClass(Void):Void {
		trace (":: testGetClass");
		var info:ClassInfo = clazz.getClass();
		assertEquals(
		"The name of the basic class changed",
		info.getName(),
		"BasicClass");
		assertEquals(
		"Problems evaluating the full name",
		info.getFullName(),
		"org.as2lib.core.BasicClass");
		trace ("------------------");
	}
}
\end{lstlisting}


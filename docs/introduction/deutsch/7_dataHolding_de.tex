\chapter{Data Holding}
\label{sec:DataHolding}

\paragraph{Beweggrund:}
Ein typisches Problem bei der Programmierung mit {\sl Actionscript 2} ist das einige Datentypen (zB.: Array) unterschiedliche Datentypen (zB.: \emph{String} und \emph{Number}) beinhalten k�nnen. Diese nicht strikte Notation kann vor allem bei einer Arbeit in Teams leicht zu Problemen f�hren.

\paragraph{L�sungsansatz:}
In der {\sl as2lib} gibt es nicht nur eine Array Klasse(TypedArray\footnote{org.as2lib.data.holder.TypedArray}), die nur bestimmte Datentypen zul�sst, sondern auch eine Vielzahl anderer Datentypen zur Datenhaltung.

\paragraph{TypedArray:} Die Klasse TypedArray erlaubt eine strikte Datentypisierung eines Arrays.
\begin{lstlisting}[frame=single]
import org.as2lib.data.holder.TypedArray;

var myA:TypedArray = new TypedArray(Number);
myA.push(2);
myA.push("Hallo");
\end{lstlisting}
In dem Codebeispiel wird ein Array vom Typ Number erstellt. Versucht ein Entwickler einen String zum TypedArray hinzuzuf�gen(\emph{myA.push}("`\textit{Hallo}"')) wirft der Compiler einen Fehler.
\begin{verbatim}
** FatalLevel ** 
org.as2lib.env.except.IllegalArgumentException: 
Type mismatch between [Hello] and [[type Function]].
  at TypedArray.validate(Hello)
\end{verbatim}
Zus�tzlich zum Typ des Arrays kann dem TypedArray Konstruktor ein bereits bestehendes Array als zweiter �bergabeparameter mit �bergeben werden. TypedArray besitzt die gleichen Funktionali�ten wie die normale Array Klasse von {\sl Macromedia}.

Weitere {\sl as2lib} Datentypen sind:
\begin{itemize}
	\item \textbf{HashMap}: Ein Datentyp dem ein Key und der dazugeh�rige Wert �bergeben werden kann. Er besitzt alle Methoden einer normalen HashMap(siehe Java).
	
\begin{lstlisting}[frame=single]
import org.as2lib.data.holder.HashMap;

var aPerson:Person = new Person("Christoph",
	"Atteneder");
var bPerson:Person = new Person("Martin",
	"Heidegger");

var nickNames:HashMap = new HashMap();
nickNames.put(aPerson,"ripcurlx");
nickNames.put(bPerson,"mastaKaneda");

trace(nickNames.get(aPerson));
trace(nickNames.get(bPerson));
\end{lstlisting}
\emph{Ausgabe:}
\begin{verbatim}
ripcurlx
mastaKaneda
\end{verbatim}
	\item \textbf{Stack}: Einem Stack k�nnen mit der Methode \emph{push} Werte hinzugef�gt bzw. mit \emph{pop} wieder entfernt werden. Es kann immer nur auf das oberste Element zugegriffen werden.
	
\begin{lstlisting}[frame=single]
import org.as2lib.data.holder.SimpleStack;

var myS:SimpleStack = new SimpleStack();
myS.push("gehts?!");
myS.push("wie");
myS.push("Hallo");
trace(myS.pop());
trace(myS.pop());
trace(myS.pop());
\end{lstlisting}
\emph{Output:}
\begin{verbatim}
Hallo
wie
gehts?!
\end{verbatim}
\item \textbf{Queue}: Einer Queue k�nnen mit der Methode \emph{enqueue} Werte hinzugef�gt bzw. mit \emph{dequeue} wieder entfernt werden. Es kann immer nur auf das erste Element zugegriffen werden. Mit der Methode \emph{peek} kann zus�tzlich auf das erste Element zugegriffen werden, ohne es aus der Reihe zu entfernen.
	
\begin{lstlisting}[frame=single]
import org.as2lib.data.holder.LinearQueue;

var aLQ:LinearQueue = new LinearQueue();
aLQ.enqueue("Hallo");
aLQ.enqueue("wie");
aLQ.enqueue("gehts?!");
trace(aLQ.peek());
trace(aLQ.dequeue());
trace(aLQ.dequeue());
trace(aLQ.dequeue());
\end{lstlisting}
\emph{Output:}
\begin{verbatim}
Hallo
Hallo
wie
gehts?!
\end{verbatim}
\end{itemize}

Zus�tzlich zu den Datentypen wurden auch \emph{Iteratoren}\footnote{Ein Iterator erm�glicht den Zugriff auf die Elemente einer Sammlung ohne Kenntnis der Struktur der Sammlung.} implementiert:

\begin{itemize}
	\item \textbf{ArrayIterator}: Da die zus�tzlichen \emph{as2lib} Datentypen intern mit Arrays aufgebaut sind, wird bei allen speziellen Iteratoren indirekt der ArrayIterator verwendet.
	\item \textbf{MapIterator}: Wenn die Methode \emph{iterator()} in einer HashMap aufgerufen wird, wird automatisch ein MapIterator zur�ckgeworfen.
\end{itemize}

M�chte man z.B.: alle Elemente einer HashMap ausgeben, kann diese Aufgabe mit einem MapIterator leicht gel�st werden.
\clearpage
\begin{lstlisting}[frame=single]
import org.as2lib.data.holder.HashMap;

var aPerson:Person = new Person("Christoph",
	"Atteneder");
var bPerson:Person = new Person("Martin",
	"Heidegger");

var nickNames:HashMap = new HashMap();
nickNames.put(aPerson,"ripcurlx");
nickNames.put(bPerson,"mastaKaneda");

var it:Iterator = myH.getIterator();

while(it.hasNext()){
	trace(it.next());
}
\end{lstlisting}
\emph{Ausgabe:}
\begin{verbatim}
Christoph,Atteneder
Martin,Heidegger
\end{verbatim}
\begin{figure}[!ht]
\begin{center}
\includegraphics{dataholder.eps}
\caption{Datenhalter des \emph{holder} Package}
\label{fig:as2libdataholder}
\end{center}
\end{figure}

Die Abbildung \ref{fig:as2libdataholder} auf Seite \pageref{fig:as2libdataholder} zeigt eine hierarchische Auflistung der Datenhalter des \emph{holder} Packages.

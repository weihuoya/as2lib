\chapter{Overloading}
\label{sec:Overloading}

\paragraph{Beweggrund:}

Da �berladen von Methoden in Java unterst�tzt wird k�nnen zB.: in einer Klasse zwei Konstruktoren vorhanden sind, die sich nur durch in ihren �bergabeparameter unterscheiden. Versucht man in \emph{Actionscript 2} mehrere Konstruktoren oder Methoden mit dem gleichen Namen und unterschiedlichen �bergabeparametern zu implementieren wird folgende Fehlermeldung ausgegeben:
\begin{quote}
	"`Eine Klasse kann nur einen Konstruktor haben!"'\\
	bzw. \\
	"`Ein Mitgliedsname darf nur einmal vorkommen!"'
\end{quote}

\paragraph{L�sungsansatz:}

Die {\sl as2lib} erm�glicht das Overloading in {\sl ActionScript 2} durch das \emph{overload}\footnote{org.as2lib.env.overload} Package. Ben�tigt man z.B.: drei Konstruktoren in einer Klasse kann dieses Problem mit der {\sl as2lib} einfach gel�st werden.

\begin{lstlisting}[frame=single]
import org.as2lib.env.overload.Overload;

class TryOverload {
	private var aString:String;
	private var aNumber:Number;
	
	public function TryOverload(){
		var overload:Overload = new Overload(this);
		overload.addHandler([Number], 
		setNumber);
		overload.addHandler([String],
		setString);
		overload.forward(arguments);
	}
	public function setNumber(aNumber:Number){
		this.aNumber = aNumber;
	}
	public function setString(aString:String){
		this.aString = aString;
	}
}
\end{lstlisting}

In diesem Codebeispiel wird in der Klasse \emph{TryOverload} ein Konstruktor erstellt, der keine bestimmten �bergabeparameter definiert. Wird eine Instanz der Klasse \emph{TryOverload} erstellt, wird zus�tlich eine Overload Objekt erzeugt, dem f�r jeden zust�tzlichen Konstructor ein eigener Handler(Typ des �bergabeparameters und aufzurufende Methode) �bergeben werden muss. In diesem speziellen Fall gibt es einen Konstruktor f�r eine Number und einen Konstruktor f�r einen String. Schlussendlich werden die �bergebenen Argumente(arguments) dem Overload Objekt �bergeben, der die entsprechende Funktion aufruft.
Ein Test der TryOverload Klasse w�rde wie folgt aussehen:

\begin{lstlisting}[frame=single]
var aOverload:TryOverload = new TryOverload("Hallo");
var bOverload:TryOverload = new TryOverload(6);
\end{lstlisting}
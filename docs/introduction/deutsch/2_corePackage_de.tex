\chapter{Core Package}
\label{sec:CorePackage}

\paragraph{Beweggrund:}
Das Festlegen und zur Verf�gung stellen gewisser Funktionalit�ten in allen Klassen vereinfacht die Entwicklung und Fehlerbehebung w�hrend der Entwicklung.

\paragraph{L�sungsansatz:}
Alle Klassen, Interfaces und Packages der {\sl as2lib} unterliegen gewissen Vorgaben. Die wichtigsten Kernklassen befinden sich im \emph{\textbf{core}-Package}.

\section{BasicInterface}
\label{sec:BasicInterface}

Zur besseren Definition von Klassenfunktionalit�ten wird in der {\sl as2lib} intensiv von \emph{Interfaces} Gebrauch gemacht. Jedes erstellte Interface der {\sl as2lib} erweitert das \emph{BasicInterface}, um folgende Funktionalit�t in jeder {\sl as2lib} Klasse sicherzustellen:

\begin{itemize}
	\item \textbf{getClass()}:\textit{ClassInfo} - Diese Methode liefert genauere Informationen zur Klasse, in der diese Funktion aufgerufen wird. Die zur�ckgegebene Information ist vom Typ \emph{ClassInfo} und beinhaltet zus�tzlich deren Klassennamen, Informationen wie Methoden, Eigenschaften, Klassenpfad und Superklasse.
	\item \textbf{toString()}:\textit{String} - Diese Methode gibt einen String der die Klasse repr�sentiert zur�ck.
\end{itemize}

Die Logik der \emph{getClass} Methode wird in der \emph{BasicClass} Klasse zur Verf�gung gestellt (siehe Abb.\  \ref{fig:corepackage}, S.\ \pageref{fig:corepackage}).

\begin{figure}[!ht]
\begin{center}
\includegraphics{core.eps}
\caption{Hauptbestandteile des \emph{core} Packages}
\label{fig:corepackage}
\end{center}
\end{figure}

\section{BasicClass}
\label{sec:BasicClass}
Die Grundklasse der {\sl as2lib} ist die \emph{BasicClass} Klasse. Alle Klassen der {\sl as2lib} sind direkt oder indirekt von der \emph{BasicClass} Klasse abgeleitet. Sie implementiert das \emph{BasicInterface} und stellt �ber die \emph{ReflectUtil}\footnote{org.as2lib.env.util.ReflectUtil} Klasse und die \emph{ObjectUtil}\footnote{org.as2lib.util.ObjectUtil} Klasse  die Logik f�r folgende Methoden zur Verf�gung:

\begin{itemize}
	\item \textbf{getClass()}:\textit{ClassInfo} - Erkl�rung siehe BasicInterface. Das Erstellen der Klasseninformationen wird durch das {\sl as2lib} \emph{reflection} Package, siehe Kapitel \ref{sec:Reflections}, erm�glicht.
	\item \textbf{toString()}:\textit{String} - Diese Methode liefert eine Darstellung der Klasse, vom Typ String, zur�ck. 
\end{itemize}
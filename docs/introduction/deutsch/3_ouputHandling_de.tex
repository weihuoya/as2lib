\chapter{Output Handling}
\label{sec:OutputHandling}

\paragraph{Beweggrund:}
Die normale Ausgabe von Applikationen in Flash wird �ber den internen Befehl
\begin{lstlisting}[frame=single]
trace(ausdruck);
\end{lstlisting}
durchgef�hrt. Die trace Ausgabe ist jedoch nur in Entwicklungsumgebungen m�glich, die den Befehl auch unterst�tzen. Bei allen anderen F�llen (z.B.: in einer Webapplikation) ist keine Standardausgabe definiert. Eine Library sollte eine standardisierte Ausgabe sowohl f�r den User als auch f�r den Entwickler zu Verf�gung stellen und �berall m�glich sein.

\paragraph{L�sungsansatz:}
Um in jeder Laufzeitumgebung\footnote{Flash kann im Player, in {\sl Macromedia Central} oder in einer kompilierten Applikation(*.exe) laufen} eine oder mehrere Ausgabem�glichkeiten zu haben, wird die \emph{Out}\footnote{org.as2lib.env.out.Out} Klasse verwendet. So kann z.B.: eine Webapplikation Fehlermeldungen auch serverseitig abspeichern, um sicherzustellen das Fehler sich nicht nur beim Kunden bemerkbar machen. Die \emph{Out} Klasse behandelt alle eingehenden Anfragen ("`Ausgabeversuche"') und leitet sie in Abh�ngigkeit von der Konfiguration an einen, oder mehrere \emph{OutputHandler}\footnote{org.as2lib.env.out.OutputHandler, org.as2lib.env.out.handler.} weiter. Die {\sl as2lib} bietet eine standardisierte Ausgabem�glichkeit f�r beliebig viele Schnittstellen.

\paragraph{Anwendung:}

Ein einfacher Anwendungsfall des {\sl as2lib} Output Handling wird in Abbildung \ref{fig:outHandling} auf Seite \pageref{fig:outHandling} dargestellt. Nachdem eine Instanz der \emph{Out} Klasse erstellt und der zur Verf�gung stehende TraceHandler\footnote{org.as2lib.env.out.handler.TraceHandler} hinzugef�gt wurde kann eine Ausgabe erfolgen. Der Ablauf der einzelnen Aktionen entspricht der Reihenfolge ihrer Nummerierung.

\begin{figure}[!ht]
\begin{center}
\includegraphics{out.eps}
\caption{Anwendungsfall des {\sl as2lib} Output Handlings}
\label{fig:outHandling}
\end{center}
\end{figure}

Es kann wie in Abbildung \ref{fig:outHandling} auf bereits definierte Ausgaben wie den \emph{TraceHandler} zugegriffen oder ein eigener \emph{OutHandler} erstellt werden(z.B.: eine Ausgabe �ber die {\sl Macromedia} Alert Komponente)\footnote{Die Macromedia Alert Komponente muss sich in der Bibliothek befinden und Sie m�ssen Flash MX 2004 Professional besitzen, um �ber die Alert Klasse darauf zugreifen zu k�nnen.}:

\begin{lstlisting}[frame=single]
import org.as2lib.env.event.EventInfo;
import org.as2lib.env.out.OutHandler;
import org.as2lib.env.out.OutInfo;
import org.as2lib.env.out.OutConfig;
import org.as2lib.core.BasicClass;
import mx.controls.Alert;

class test.org.as2lib.env.out.handler.UIAlertHandler 
	extends BasicClass implements OutHandler {

	public function write(info:OutInfo):Void {
		Alert.show(info.getMessage(), 
			getClass().getName());
	}

}
\end{lstlisting}

Durch die Festlegung von Ausgabe Levels(z.B.: \emph{out.setLevel(Out.DEBUG)}) kann die Ausgabe bestimmter Informationen verhindert werden und unterschiedliche Ausgabe Levels verwenden.
Die m�glichen Ausgabe Levels sind:
\begin{itemize}
	\item Out.ALL
	\item Out.DEBUG
	\item Out.INFO
	\item Out.WARNING
	\item Out.ERROR
	\item Out.FATAL
	\item Out.NONE
\end{itemize}

Diese Abstufung erm�glicht einerseits ein �bersichtlicheres Debuggen bei der Entwicklung, als auch ein schnelleres Umschreiben der Ausgabe in einer fertigen Applikation.
Verwendet man bei der Entwicklung z.B.: \emph{Out.DEBUG} werden bei dieser Konfiguration alle Informationen die sich in einem niedrigeren Level, als das gesetzte(\textit{DEBUG}), befinden ausgegeben: \textit{DEBUG}, \textit{INFO}, \textit{WARNING}, \textit{ERROR}, \textit{FATAL}.
Nur die \textit{LOG} Ausgabe wird unterdr�ckt.

\begin{lstlisting}[frame=single]
var out:Out = new Out();
out.addHandler(new TraceHandler());

out.setLevel(Out.DEBUG);

out.log("log mich bitte!");
out.debug("debug mich bitte!");
out.info("informiere mich bitte!");
out.warning("warne mich bitte!");
out.error(new Exception("Output Error", this,
  arguments));
out.fatal(new FatalException("Fatal Output Error",
	this, arguments));
\end{lstlisting}

M�chte man in der fertigen Applikation nur die schwerwiegenden Fehler ausgeben, l�sst sich das durch eine einfache Zeile in der Applikation bewerkstelligen.

\begin{lstlisting}[frame=single]
aOut.setLevel(Out.FATAL);
\end{lstlisting}

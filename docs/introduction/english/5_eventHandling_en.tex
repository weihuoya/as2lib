\chapter{Event Handling}
\label{sec:EventHandling}

\paragraph{Motivation:}
Events are pretty performance intensive in {\sl Flash} and are a fundamental part of user interface development.
Most developers use the intrinsic \emph{AsBroadcaster}(an undocumented feature of Flash) or \emph{EventDispatcher}\footnote{mx.events.EventDispatcher} classes of {\sl Macromedia} for this functionality. Exact definitions aren't available for \emph{EventListener} nor for single events or for arguments. Some information and definitions of listeners for developers are also missing.

\paragraph{Solution:}

Problems to cope with:

\begin{itemize}
	\item Object developer has to define which events can be caught.
	\item Listener developer has to implement all events.
	\item Object developer has to be able to dispatch events.
	\item Object developer should be able to add information to a specific event.
\end{itemize}

The {\sl as2lib} supports \emph{Event Handling} because it�s a core part of application development. If you use the Flash intrinsic \emph{AsBroadcaster} it can result in an inefficient implementation. The {\sl Macromedia} \emph{EventDispatcher} isn�t free, i.e. is only available when you have purchased Macromedia Flash, and doesn�t support all of the required functionality.

The most important interface of the \emph{event} packages a developer gets in touch with is the \emph{EventBroadcaster}\footnote{org.as2lib.env.event.EventBroadcaster}. You can add as many listeners as you like,
\begin{lstlisting}[frame=single]
addListener(listener:EventListener)
\end{lstlisting}
and remove them if you wish to.
\begin{lstlisting}[frame=single]
removeListener(listener:EventListener)
\end{lstlisting}

Listener objects have to implement the {\sl EventListener} Interface. For your own projects it is recommended to implement and use your own EventListener interface.

If you like to use the \emph{EventBroadcaster} just do it the following way. The figure \ref{fig:as2libeventhandling} on page \pageref{fig:as2libeventhandling} shows a simple example application of {\sl as2lib} Event Handling. Aside from the SimpleEventListener only {\sl as2lib} classes are used.

\begin{lstlisting}[frame=single][caption=SimpleEventListener]
import org.as2lib.env.event.EventListener;
import org.as2lib.core.BasicClass;

class com.myproject.SimpleEventListener 
	extends BasicClass implements EventListener {
	
	public function onTest(){
		trace("onTest");
	}
}
\end{lstlisting}

\begin{figure}
\begin{center}
\includegraphics{uml/as2lib/eventhandling.eps}
\caption{Process of an {\sl as2lib} Event Handling example.}
\label{fig:as2libeventhandling}
\end{center}
\end{figure}

\chapter{Test Cases}
\label{sec:TestCases}

A \emph{Test Case} is a class/method that verifies the correct behaviour of a specific class/method. Every developer has to create test cases for his or her own classes and is assisted by a test case API that offers functionality for automated testing.

\paragraph{Motivation:}
Although two testing API's already exists for {\sl Actionscript 2}, as2unit\footnote{\href{http://www.as2unit.org}{asunit.sourceforge.net}} and asunit\footnote{\href{http://asunit.sourceforge.net}{www.asunit.org}}, our own testing API was created because the following reasons:

\begin{itemize}
	\item {\sl As2unit} supports only few functionalities (as2unit 7 test methods - as2lib 17 test methods).
	\item {\sl As2unit} is in spite of the disclosure of the source code still not open source.
	\item Official documentation of {\sl as2unit} is still not available.
	\item {\sl As2unit} is only available as a component.
	\item {\sl As2unit} can always test only one class.
	\item {\sl As2unit} is not build to support different views. 
	\item {\sl As2unit} does not support pause() and resume() during TestCases (Tests that have a time delay like using onEnterframe). 
\end{itemize} 

\paragraph{Solution:}

Two actions shouldn't be necessary (including the Flash component, setting the parameters) to perform test cases, as it is presently the case with {\sl as2unit}. It should be possible for the developer to execute a test case through only one method call. A direct call of a class as well as a call of a whole package is possible. 
\clearpage
\begin{lstlisting}[frame=single]
// Use your TestCase here:
new test.org.as2lib.core.TReflections().run();
\end{lstlisting}

If you have more than one TestCase you can use a simplified way:
\begin{lstlisting}[frame=single]
import org.as2lib.test.unit.TestSuiteFactory;
import test.org.as2lib.core.* // Import Statement, to reduce the Lines.

// Usage to classes to include them in compile time.
TestReflections;
TestBasicClass;
TestEventHandling;

new TestSuiteFactory().collectAllTestCases().run();
\end{lstlisting}


The cases to test must be invoked before the test call so that they are available while run time.
Similar to test case APIs like: JUnit\footnote{\href{http://www.junit.org}{www.junit.org}}, a variety of methods are available for the developer (optional parameters are marked with []):

\begin{itemize}
	\item \textbf{assertTrue}([\textit{message:String}], \textit{testVar1:Boolean}) - An error message will be reported if testVar1 is false.
	\item \textbf{assertFalse}([\textit{message:String}],\textit{ testVar1:Boolean}) - An error message will be reported if testVar1 is true.
	\item \textbf{assertEquals}([\textit{message:String}], \textit{testVar1}, \textit{testVar2}) - An error message will be reported if the passed parameters are not equal (!=).
	\item \textbf{assertNotEquals}([\textit{message:String}], \textit{testVar1}, \textit{testVar2}) - An error message will be reported if the passed parameters are equal (==).
	\item \textbf{assertSame}([\textit{message:String}], \textit{testVar1}, \textit{testVar2}) - An error message will be reported if the passed parameters are not the same object (!==).
	\item \textbf{assertNotSame}([\textit{message:String}], \textit{testVar1}, \textit{testVar2}) - An error message will be reported if the passed parameters are the same object (===).
	\item \textbf{assertNull}([\textit{message:String}], \textit{testVar1}) - An error message will be reported if testVar1 is not null.
	\item \textbf{assertNotNull}([\textit{message:String}], \textit{testVar1}) - Error message if testVar1 is null.
	\item \textbf{assertUndefined}([\textit{message:String}], \textit{testVar1}) - Error message if testVar1 is not undefined.
	\item \textbf{assertNotUndefined}([\textit{message:String}], \textit{testVar1}) - Error message if testVar1 is undefined.
	\item \textbf{assertIsEmpty}([\textit{message:String}], \textit{testVar1}) - If testVar1 is neither undefined nor null an error message will be reported.
	\item \textbf{assertIsNotEmpty}([\textit{message:String}], \textit{testVar1}) - If testVar1 is undefined or null an error message will be reported.
	\item \textbf{assertInfinity}([\textit{message:String}], \textit{testVar1}) - Error message if testVar1 is not infinite.
	\item \textbf{assertNotInfinity}([\textit{message:String}], \textit{testVar1}) - Error message if testVar1 is infinite.
	\item \textbf{fail}(\textit{message:String}) - Adds a custom failure message to the whole error output.
	\item \textbf{assertThrows}(\textit{exception:Function, atObject, theFunction:String, parameter:Array}) - If no exception is thrown during the execution of the passed function(theFunction) of the object(atObjekt) with the parameter values(parameter) an error message will be reported.
	\item \textbf{assertNotThrows}(\textit{exception:Function, atObject, theFunction:String, parameter:Array}) - If an exception is thrown during the execution of the passed function(theFunction) of the object(atObjekt) with the parameter values(parameter) an error message will be reported.
\end{itemize}

A test case must extend the class TestCase\footnote{org.as2lib.test.unit.TestCase}. All methods of the test case that start with ``test'' are executed by the testing API. The following coding example should show you how to use the Test Case API.
\begin{lstlisting}[frame=single]
import org.as2lib.test.unit.TestCase;
import org.as2lib.core.BasicClass;
import org.as2lib.env.reflect.ClassInfo;

class test.org.as2lib.core.TReflections extends TestCase {
	private var clazz:BasicClass;
	
	public function TReflections(Void) {}
	
	public function setUp(Void):Void {
		clazz = new BasicClass();
	}
	
	public function testGetClass(Void):Void {
		var info:ClassInfo = clazz.getClass();
		
		assertEquals(
			"The name of the basic class changed",
			info.getName(),
			"BasicClass");
		
		assertEquals(
			"Problems evaluating the full name",
			info.getFullName(),
			"org.as2lib.core.BasicClass");
	}
	
	public function tearDown(Void):Void {
		delete clazz;
	}
}
\end{lstlisting}



\chapter{Core Package}
\label{sec:CorePackage}

\paragraph{Motivation:}
Fournir des fonctionnalit�s sp�cifiques dans toutes les classes afin de simplifier le d�veloppement et la r�paration de bug.

\paragraph{Solution:}
Toutes les classes, interfaces et packages d'as2lib suivent le m�me guide. Les classes de fondation sont dans le package \emph{\textbf{core}}\footnote{org.as2lib.core.*}.

\section{BasicInterface}
\label{sec:BasicInterface}

Afin d'obtenir une meilleure d�finition des classes, les fonctionnalit�s d'{\sl as2lib} sont d�finies sous forme d'\emph{interface} de fa�on intensive. Toutes les interfaces d'{\sl as2lib} h�ritent de l'interface \emph{BasicInterface}, ce qui assure d'avoir ces fonctionnalit�s dans toutes les classes:

\begin{itemize}
	\item \textbf{getClass()}:\textit{ClassInfo} - Cette m�thode offre l'information exacte sur la classe au moment o� la m�thode est invoqu�.	L'information retourn�e est de type \emph{ClassInfo} et contient le nom de la classe, m�thodes, propri�t�s, le nom complet incluant le package et les classes M�res.
	\item \textbf{toString()}:\textit{String} - Cette m�thode retourne la repr�sentation de l'objet sous forme de String.
\end{itemize}

La logique de la m�thode getClass est fournie par la classe \emph{BasicClass} (voir fig.\  \ref{fig:as2libcorepackage}, S.\ \pageref{fig:as2libcorepackage}).

\begin{figure}[!ht]
\begin{center}
\includegraphics{uml/as2lib/core.eps}
\caption{Partie principale du package \emph{org.as2lib.core}}
\label{fig:as2libcorepackage}
\end{center}
\end{figure}

\section{BasicClass}
\label{sec:BasicClass}

La classe de base d'{\sl as2lib} est \emph{BasicClass}. Toutes les classes d'{\sl as2lib} sont directement ou indirectement d�riv�es de \emph{BasicClass}.  Elle impl�mente \emph{BasicInterface} et fournie,  � l'aide des classes \emph{ReflectUtil}\footnote{org.as2lib.env.util.ReflectUtil} et \emph{ObjectUtil}\footnote{org.as2lib.util.ObjectUtil}, la logique de ces m�thodes qui doivent �tre impl�ment�:

\begin{itemize}
	\item \textbf{getClass()}:\textit{ClassInfo} - Pour la d�finition, voir la documentation de \emph{BasicInterface}. La cr�ation de l'information d'une classe est possible gr�ce au package reflection d'{\sl as2lib} (org.as2lib.env.reflect) voir \ref{sec:Reflections}.
	\item \textbf{toString()}:\textit{String} - Cette m�thode retourne la repr�sentation de l'objet sous forme de String.
\end{itemize}
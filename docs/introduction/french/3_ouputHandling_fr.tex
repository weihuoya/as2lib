\chapter{Output Handling}
\label{sec:OutputHandling}

\paragraph{Motivation:}
La sortie normale d'une application Flash est fait via l'op�ration interne:
\begin{lstlisting}[frame=single]
trace(expression);
\end{lstlisting}
La sortie d'un trace est n�anmoins seulement visible qu'� l'int�rieur d'un environnement de d�veloppement qui supporte cette op�ration.  Dans tous les autres cas (ex: dans un application Web), il n'y a pas de standard de d�fini. Une librairie devrait fournir une sortie standardis�e pour les utilisateurs et les d�veloppeurs dans tous les cas possibles.

\paragraph{Solution:}
Pour avoir plusieurs types de sortie dans tous les environnements\footnote{Flash peut rouler dans le player, dans {\sl Macromedia Central}, dans un application compil� (*.exe)}, il suffit d'utiliser la classe \emph{Out}\footnote{org.as2lib.env.out.Out}. Cela donne la possibilit� par exemple de sauvegarder les messages d'erreur c�t� serveur, afin de s'assurer que l'erreur soit perceptible seulement pour le d�veloppeur et non pour le client. La classe \emph{Out} traite les requ�tes selon la configuration de un ou plusieurs \emph{OutputHandler}\footnote{org.as2lib.env.out.OutputHandler, org.as2lib.env.out.handler.*}. {\sl As2lib} offre une sortie standardis�e pour un nombre illimit� d'interfaces.

\paragraph{Usage:}
Un exemple simple de l'utilisation de la gestion de sortie d'{\sl as2lib} est vu dans la figure \ref{fig:outHandling},p. \pageref{fig:outHandling}. Apr�s avoir cr�� une instance de la classe \emph{Out} et fournie le \emph{TraceHandler}\footnote{org.as2lib.env.out.handler.TraceHandler}, on peut proc�d� � l'utilisation.

\begin{figure}[!ht]
\begin{center}
\includegraphics{uml/as2lib/out.eps}
\caption{Utilisation de la gestion de sortie d'{\sl as2lib}}
\label{fig:outHandling}
\end{center}
\end{figure}

La s�quence de chaque action est d�termin�e par leurs num�ros. 
	On peut, comme dans la figure \ref{fig:outHandling}, p. \pageref{fig:outHandling}, utilis� des sorties pr�d�fini comme le \emph{TraceHandler} ou \emph{ExternalConsoleHandler}\footnote{org.as2lib.env.out.handler.ExternalConsoleHandler} ou bien faire sa propre impl�mentation de l'interface \emph{OutputHandler}(ex.: sortie avec le component Alert de Macromedia)\footnote{Le component Alert de {\sl Macromedia} doit �tre dans la librairie et vous devez poss�der {\sl Flash MX 2004 Professional}, pour acc�der � la classe Alert.}:

\begin{lstlisting}[frame=single]
import org.as2lib.env.event.EventInfo;
import org.as2lib.env.out.OutHandler;
import org.as2lib.env.out.info.OutWriteInfo;
import org.as2lib.env.out.info.OutErrorInfo;
import org.as2lib.env.out.OutConfig;
import org.as2lib.core.BasicClass;
import mx.controls.Alert;

class test.org.as2lib.env.out.handler.UIAlertHandler 
	extends BasicClass implements OutHandler {

	public function write(info:OutWriteInfo):Void {
		Alert.show(info.getMessage(), 
			getClass().getName());
	}
	
	public function error(info:OutErrorInfo):Void {
		Alert.show(
			OutConfig.getErrorStringifier().execute(info),
			getClass().getName()
		);
	}
}
\end{lstlisting}

Par la d�finition d'un niveau de sortie (ex.: \emph{aOut.setLevel(Out.DEBUG)}), il est possible d'emp�cher la sortie de certaines informations.
Les niveaux possibles sont :
\begin{itemize}
	\item Out.ALL
	\item Out.DEBUG
	\item Out.INFO
	\item Out.WARNING
	\item Out.ERROR
	\item Out.FATAL
	\item Out.NONE
\end{itemize}

Cette graduation permet un d�buguage clair durant le d�veloppement et une r��criture rapide de la sortie lorsque l'application est termin�e.  Durant le d�veloppement (ex.: avec \emph{DEBUG}), toutes les informations ayant un niveau inf�rieur � celui s�lectionn� sont utilis�s (\textit{DEBUG}): \textit{DEBUG}, \textit{INFO}, \textit{WARNING}, \textit{ERROR} et \textit{FATAL}. Il n'y a que \textit{LOG} qui est omis.

\begin{lstlisting}[frame=single]
var aOut = new Out();

aOut.setLevel(Out.DEBUG);

aOut.log("log me Please!");
aOut.debug("debug me Please!");
aOut.info("inform me Please!");
aOut.warning("warn me Please!");
aOut.error(new Exception("Output Error", this));
aOut.fatal(new FatalException("Fatal Output Error",
	this));
\end{lstlisting}

Lorsque l'application est finie, sortir seulement les erreurs fatales peut �tre fait avec une seule ligne de code.

\begin{lstlisting}[frame=single]
aOut.setLevel(Out.FATAL);
\end{lstlisting}

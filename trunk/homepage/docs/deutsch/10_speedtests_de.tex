\chapter{Speed Tests}
\label{sec:SpeedTests}

\paragraph{Beweggrund:}
Das Testen von Applikationen auf dessen Performanz ist unumg�nglich, da diese meistens direkt mit dem Erfolg der Applikation zusammenh�ngt.

\paragraph{L�sungsansatz:}
Die Durchf�hrung eines Speed Testes l�sst sich nach folgendem Schema durchf�hren:
\begin{itemize}
	\item Erstellen der einzelnen Testf�lle.
	Beispiel f�r MyAddSpeedTest:
\begin{lstlisting}[frame=single]
import org.as2lib.test.speed.TestCase;

class MyAddSpeedTest implements TestCase {
	private var a:Number = 0;
	public function run (Void):Void {
		a++;
	}
}
\end{lstlisting}
	\item Importieren der SpeedTest und der Config Klasse.
\begin{lstlisting}[frame=single]
import org.as2lib.test.speed.Test;
import org.as2lib.Config;
\end{lstlisting}
	\item Erstellen einer Instanz der Test Klasse und �bergabe des Output Handlers.
\begin{lstlisting}[frame=single]
var test:Test = new Test();
test.setOut(Config.getOut());
\end{lstlisting}
	\item Setzen der Anzahl der durchzuf�hrenden Test F�lle.
\begin{lstlisting}[frame=single]
test.setCalls(1000);
\end{lstlisting}
	\item Nachdem die durchzuf�hrenden Testf�lle hinzugef�gt worden sind kann der Test durchgef�hrt werden. Der �bergabeparameter \emph{true} in \emph{test.run(}) bewirkt eine sofortig Ausgabe des Testergebnisses.
\begin{lstlisting}[frame=single]
test.addTestCase(new MyAddSpeedTest());
test.addTestCase(new MyMinusSpeedTest());
test.run(true);
\end{lstlisting}
\end{itemize} 

\emph{Ausgabe:}
\begin{verbatim}
** InfoLevel ** 
	
 * Testresult [1000 calls] * 
 116\% MyAddSpeedTest: total time:29ms; 
	calls/second:34482; 
	average time:0.029ms; (+0.004ms)
[fastest] 100\% MyMinusSpeedTest: total time:25ms;
	calls/second:40000;
	average time:0.025ms; 
\end{verbatim}Die Ausgabe des SpeedTests zeigt die Anzahl der Aufrufe, die verstrichene Gesamtzeit und den durchschnittlichen Zeitbedarf. Sie findet den schnellsten Testfall und gibt das prozentuelle Verhalten der anderen Testf�lle im Verh�ltnis zum Schnellsten aus.

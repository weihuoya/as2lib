\chapter{Reflections}
\label{sec:Reflections}

\paragraph{Beweggrund:}

In Java ist es m�glich Informationen �ber den Namen der Klasse, deren Methoden und Eigenschaften �ber Reflections zu ermitteln. Diese eingebaute Funktionalit�t von Java ist in {\sl Flash} bzw. {\sl Actionscript 2} nicht integriert und wird deshalb von der {\sl as2lib} unterst�tzt.

\paragraph{L�sungsansatz:}
Um diese Funktionalit�ten in {\sl Flash} verwenden zu k�nnen wurden die Reflections nach folgendem Schema implementiert.
Ausgehend vom Namensbereich \emph{\_global} wird die Actionscript Klassenstruktur durchsucht. Wird ein Objekt gefunden, wird es als Package erkannt. Ein Unterobjekt vom Typ Function wird als Klasse gekennzeichntet.

Eine der vielen Anwendungen von Reflections in der {\sl as2lib} ist zB.: die \emph{BasicClass}, siehe Kapitel \ref{sec:CorePackage}. Die Methode \emph{getClass}, der BasicClass greift auf die Methode \emph{getClassInfo} der ReflectUtil\footnote{org.as2lib.env.util.ReflectUtil} Klasse zu und gibt ein \emph{ClassInfo} Objekt zur�ck, dass alle wichtigen Informationen der Klasse zur Verf�gung stellt. Das \emph{reflect}\footnote{org.as2lib.env.reflect} Package arbeitet mit unterschiedlichen Algorithmen um an die Klasseninformationen zu gelangen. Die Sammlung aller Algorithmen befindet sich im \emph{algorythm}\footnote{org.as2lib.env.reflect.algorythm} Package der \emph{as2lib}. Die Funktionali�ten der Reflections k�nnen auch direkt �ber die \emph{ReflectUtil} Klasse verwendet werden.
\begin{lstlisting}[frame=single]
import org.as2lib.env.util.ReflectUtil;
import org.as2lib.env.reflect.ClassInfo;
import edu.test.TestClass;

var aClass:TestClass = new TestClass();
var aClassInfo:ClassInfo = ReflectUtil.getClassInfo(
				aClass);
\end{lstlisting}
Das erzeugte Objekt vom Typ \emph{ClassInfo} beinhaltet Methoden um genauere Informationen der Klasse zu erhalten.
Seine Methoden sind:

\begin{itemize}
	\item \textbf{getName(}):\textit{String} - Name der Klasse z.B.: "`TestClass"'.
	\item \textbf{getFullName()}:\textit{String} - Name der Klasse inklusive Packageinformationen z.B.: "`edu.test.TestClass"'.
	\item \textbf{getRepresentedClass()}:\textit{Function} - Klasse zu der die Klasseninformation erstellt wurde.
	\item \textbf{getConstructor()}: \textit{ConstructorInfo} - Gibt Konstruktor der Klasse als ConstructorInfo zur�ck.
	\item \textbf{getSuperClass(}):\textit{ClassInfo} - Informationen zur Vaterklasse, falls vorhanden.
	\item \textbf{newInstance(}) - Neue Instanz der Klasse zu der die Klasseninformation erstellt wurde.
	\item \textbf{getParent()}:\textit{PackageInfo} - Informationen zum Package in dem sich die Klasse befindet.
	\item \textbf{getMethods()}:\textit{Map} - Map, siehe Kapitel \ref{sec:DataHolding}, die Informationen zu den einzelnen Methoden der Klasse beinhaltet.
	\item \textbf{getMethod(methodName:String)}:\textit{MethodInfo} - Gibt Information zur Methode dessen Name �bergeben wurde als MethodInfo zur�ck.
	\item \textbf{getMethod(concreteMethod:Function)}:\textit{MethodInfo} - Gibt Information zur Methode die �bergeben wurde als MethodInfo zur�ck.
	\item \textbf{getProperties()}:\textit{HashMap} - HashMap, die Informationen zu den einzelnen Eigenschaften, falls sie eine get set Methode besitzen zur�ckliefert.
	\item \textbf{getProperty(propertyName:String)}:\textit{PropertyInfo} - Gibt Information zur Eigenschaft dessen Name �bergeben wurde als PropertyInfo zur�ck.
	\item \textbf{getProperty(concreteProperty:Function)}:\textit{PropertyInfo} - Gibt Information zur Eigenschaft die �bergeben wurde als PropertyInfo zur�ck.
\end{itemize}
Die Verkn�pfungen der einzelnen Info Klassen zeigt die Abbildung \ref{fig:as2libreflect} auf Seite \pageref{fig:as2libreflect}.
\begin{figure}[!ht]
\begin{center}
\includegraphics{reflect.eps}
\caption{Hierarchie der Info Klassen im \emph{reflect} Package}
\label{fig:as2libreflect}
\end{center}
\end{figure}


\chapter{Data Holding}
\label{sec:DataHolding}

\paragraph{Motivation:}
A typical problem when using {\sl ActionScript 2} is that some data types (e.g. Array) can contain different data types (e.g. String, Number,...). This flexible notation could result, especially in teams, in misuse of these data holders.

\paragraph{Solution:}
{\sl As2lib} not only provides a strictly typed Array class(TypedArray\footnote{org.as2lib.data.holder.TypedArray}) but also a lot of other data types for data holding.

\paragraph{TypedArray:} The TypedArray class provides strict typing of an Array.
\begin{lstlisting}[frame=single]
import org.as2lib.data.holder.TypedArray;

var myA:TypedArray = new TypedArray(Number);
myA.push(2);
myA.push("Hello");
\end{lstlisting}
In this code snippet an Array of type Number is created. If you try to add a string to the TypedArray(\emph{myA.push}(``\textit{Hello}'')) the compiler throws the following exception:
\begin{verbatim}
** FatalLevel ** 
org.as2lib.env.except.IllegalArgumentException: 
Type mismatch between [Hello] and [[type Function]].
  at TypedArray.validate(Hello)
\end{verbatim}
Additionally you can pass an existing Array as second parameter value. TypedArray provides the same functionality as a regular {\sl Macromedia} Array class.

More {\sl as2lib} data types are:
\begin{itemize}
	\item \textbf{HashMap} : A data type which holds keys and their values. It has every common method of a regular HashMap (see Java).
	
\begin{lstlisting}[frame=single]
import org.as2lib.data.holder.HashMap;

var aPerson:Person = new Person("Christoph",
	"Atteneder");
var bPerson:Person = new Person("Martin",
	"Heidegger");

var nickNames:HashMap = new HashMap();
nickNames.put(aPerson,"ripcurlx");
nickNames.put(bPerson,"mastaKaneda");

trace(nickNames.get(aPerson));
trace(nickNames.get(bPerson));
\end{lstlisting}
\emph{Output:}
\begin{verbatim}
ripcurlx
mastaKaneda
\end{verbatim}

	\item \textbf{Stack} : You can add values to a Stack with the \emph{push} method and remove them with the \emph{pop} method. Only the last added value can be accessed.
	
\begin{lstlisting}[frame=single]
import org.as2lib.data.holder.SimpleStack;

var myS:SimpleStack = new SimpleStack();
myS.push("uuuuuuup?!");
myS.push("what's");
myS.push("Hi");
trace(myS.pop());
trace(myS.pop());
trace(myS.pop());
\end{lstlisting}
\emph{Output:}
\begin{verbatim}
Hi
what's
uuuuuuup?!
\end{verbatim}

\item \textbf{Queue} : In contrast to the Stack you can only access the first element. You can add values with the \emph{enqueue} method and remove them with the \emph{dequeue} method. If you don�t want to remove the first element while accessing it you can use the \emph{peek} method.

\begin{lstlisting}[frame=single]
import org.as2lib.data.holder.LinearQueue;

var aLQ:LinearQueue = new LinearQueue();
aLQ.enqueue("Hi");
aLQ.enqueue("whats");
aLQ.enqueue("uuuuup?!");
trace(aLQ.peek());
trace(aLQ.dequeue());
trace(aLQ.dequeue());
trace(aLQ.dequeue());
\end{lstlisting}
\emph{Output:}
\begin{verbatim}
Hi
Hi
whats
uuuuup?!
\end{verbatim}
\end{itemize}

In addition to these data types \emph{Iterators}\footnote{An iterator makes it easier to access elements of a collection without knowing its structure.} are also provided:

\begin{itemize}
	\item \textbf{ArrayIterator} : Because of the fact that the additional data types are based internally on Arrays, every special Iterator indirectly uses the ArrayIterator{\footnote{org.as2lib.data.io.iterator.ArrayIterator}}.
	\item \textbf{MapIterator} : If you call the \emph{iterator()} method in a HashMap it returns a MapIterator to you.
\end{itemize}
If you like to print every element of a HashMap, you can do so with a MapIterator\footnote{org.as2lib.data.iterator.MapIterator}:
\clearpage
\begin{lstlisting}[frame=single]
import org.as2lib.data.holder.HashMap;

var aPerson:Person = new Person("Christoph",
	"Atteneder");
var bPerson:Person = new Person("Martin",
	"Heidegger");

var nickNames:HashMap = new HashMap();
nickNames.put(aPerson,"ripcurlx");
nickNames.put(bPerson,"mastaKaneda");

var it:Iterator = myH.getIterator();

while(it.hasNext()){
	trace(it.next());
}
\end{lstlisting}
\emph{Output:}
\begin{verbatim}
Christoph,Atteneder
Martin,Heidegger
\end{verbatim}

\begin{figure}[!ht]
\begin{center}
\includegraphics{uml/as2lib/dataholder.eps}
\caption{data holders of \emph{holder} package}
\label{fig:as2libdataholder}
\end{center}
\end{figure}

Figure \ref{fig:as2libdataholder} on page \pageref{fig:as2libdataholder} shows a hierarchical structure of data holders in the \emph{holder} package.

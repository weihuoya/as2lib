\chapter{Reflections}
\label{sec:Reflections}

\paragraph{Motivation:}

In Java it is possible to obtain information about the name of a class, its methods and properties via \emph{Reflections}. This built-in functionality of Java is not integrated into {\sl ActionScript 2} and therefore supported by {\sl as2lib}.

\paragraph{Solution:}
To use these functionalities in {\sl Flash} the reflections were implemented after the following schema.
Starting from the "`root package"' \emph{\_global} the ActionScript classes are searched through. Is an object found, it is recognized as a package. A sub-object of type function is marked as a class. 


One usage among many for \emph{Reflections} in {\sl as2lib} is for instance the BasicClass, see chapter \ref{sec:CorePackage}. The \emph{getClass} method of the \emph{BasicClass} uses the \emph{getClassInfo} method of the \emph{ReflectUtil}\footnote{org.as2lib.env.util.ReflectUtil} class and returns a \emph{ClassInfo} instance that provides all important information about the class. The \emph{reflect}\footnote{org.as2lib.env.reflect} package uses different algorithms to get the information about the class. The collection containing all algorithms is located in the \emph{algorythm}\footnote{org.as2lib.env.reflect.algorythm} package of the \emph{as2lib}. The functionality can, of course, also be accessed directly through the \emph{ReflectUtil} class.
\begin{lstlisting}[frame=single]
import org.as2lib.env.util.ReflectUtil;
import org.as2lib.env.reflect.ClassInfo;
import edu.test.TestClass;

var aClass:TestClass = new TestClass();
var aClassInfo:ClassInfo = ReflectUtil.getClassInfo(
				aClass);
\end{lstlisting}
The created \emph{ClassInfo} instance contains methods which allow the developer to get further information about the class.
Its methods are:

\begin{itemize}
	\item \textbf{getName(}):\textit{String} - Name of the class e.g.: "`TestClass"'.
	\item \textbf{getFullName()}:\textit{String} - Name of the class including information about the package e.g.: "`edu.test.TestClass"'.
	\item \textbf{getRepresentedClass()}:\textit{Function} - Reference to the class the information is about.
	\item \textbf{getConstructor()}: \textit{ConstructorInfo} - Returns the constructor of the class wrapped into a ConstructorInfo instance.
	\item \textbf{getSuperClass(}):\textit{ClassInfo} - Informaction about the superclass, if existing.
	\item \textbf{newInstance(}) - Creates a new instance of the class the information is about.
	\item \textbf{getParent()}:\textit{PackageInfo} - Information about the package the class is located in.
	\item \textbf{getMethods()}:\textit{Map} - Map, see chapter \ref{sec:DataHolding}, containing information about every single method of the class.
	\item \textbf{getMethod(methodName:String)}:\textit{MethodInfo} - Returns information about the method whose name was passed as a ClassInfo instance.
	\item \textbf{getMethod(concreteMethod:Function)}:\textit{MethodInfo} - Returns information about the given method in form of a MethodInfo instance.
	\item \textbf{getProperties()}:\textit{HashMap} - HashMap containing information about properties of the class specified by getters and setters.
	\item \textbf{getProperty(propertyName:String)}:\textit{PropertyInfo} - Returns information about the property whose name was passed in form of a PropertyInfo instance. 
	\item \textbf{getProperty(concreteProperty:Function)}:\textit{PropertyInfo} - Returns information about the given property as a PropertyInfo instance.
\end{itemize}

\begin{figure}
\begin{center}
\includegraphics{uml/as2lib/reflect.eps}
\caption{Hierarchy of the Info Classes in the \emph{reflect} package}
\label{fig:as2libreflect}
\end{center}
\end{figure}
The connection between the Info classes is shown by the figure \ref{fig:as2libreflect} on page \pageref{fig:as2libreflect}.

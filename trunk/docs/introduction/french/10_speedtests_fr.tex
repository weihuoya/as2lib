\chapter{Speed Tests}
\label{sec:SpeedTests}

\paragraph{Motivation:}
Tester les applications pour leur performance est in�vitable parce que le succ�s d'une application d�pend largement sur la performance.

\paragraph{Solution:}
On peut proc�der � un test de vitesse en suivant le sch�ma suivant:
\begin{itemize}
	\item Creating of single test cases.
	Example for MyAddSpeedTest:
\begin{lstlisting}[frame=single]
import org.as2lib.test.speed.TestCase;

class MyAddSpeedTest implements TestCase {
	private var a:Number = 0;
	public function run (Void):Void {
		a++;
	}
}
\end{lstlisting}
	\item Import� les classes SpeedTest et OutputHandler.
\begin{lstlisting}[frame=single]
import org.as2lib.test.speed.Test;
import org.as2lib.Config;
\end{lstlisting}
	\item Cr�er une instance de la classe Test  et lui envoy� le OutputHandler.
\begin{lstlisting}[frame=single]
var test:Test = new Test();
test.setOut(Config.getOut());
\end{lstlisting}
	\item Assigner le nombre de test � �tre ex�cut�.
\begin{lstlisting}[frame=single]
test.setCalls(1000);
\end{lstlisting}
	\item Le test peut d�buter apr�s avoir ajout� les tests unitaires qui doivent �tre test�s. La param�tre \emph{true} dans \emph{test.run()} d�clanche la sortie imm�diate des r�sultats du test.
\begin{lstlisting}[frame=single]
test.addTestCase(new MyAddSpeedTest());
test.addTestCase(new MyMinusSpeedTest());
test.run(true);
\end{lstlisting}
\end{itemize}
\clearpage
\emph{Output:}
\begin{lstlisting}[frame=single]
** InfoLevel **
-- Testresult [2000 calls] --
187% TypedArrayTest: total time:457ms; 
	average time:0.2285ms; (+0.106ms)
111% ArrayTest: total time:272ms; 
	average time:0.136ms; (+0.014ms)
[fastest] 100% ASBroadcasterTest: total time:245ms; 
	average time:0.1225ms;
175% EventDispatcherTest: total time:428ms; 
	average time:0.214ms; (+0.092ms)
191% EventBroadcasterTest: total time:469ms; 
	average time:0.2345ms; (+0.112ms)
\end{lstlisting}
La sortie du test de vitesse affiche le nombre d'appels, le temps en �coul� au total et le temps �coul� en moyenne. Il trouve le test le plus rapide et affiche sous forme de pourcentage l'efficacit� de chaque test par rapport au plus rapide.